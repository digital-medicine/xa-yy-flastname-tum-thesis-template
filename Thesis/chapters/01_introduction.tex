\chapter{Introduction}

\textit{Note: Introduce the topic of your thesis, e.g. with a little historical overview.}

\section{Problem}

\textit{Note: Describe the problem that you like to address in your thesis to show the importance of your work. Focus on the negative symptoms of the currently available solution.}

\section{Proposed Solution}

\textit{Note: Motivate scientifically why solving this problem is necessary. What kind of benefits do we have by solving the problem? How do you plan on solving the problem?}

\section{Objectives}

\textit{Note: Describe the research goals and/or research questions and how you address them by summarizing what you want to achieve in your thesis, e.g. developing a system and then evaluating it.}

\section{Research Approach and Methodology}

Describe your research approach and your methods. Explain how you want to address the problem you described earlier. You can already define a first hypothesis such as: "Automated Delivery in early requirements engineering improve the amount of user involvement.".

\subsection{Research Approach} 

The following selection should help you select your research approach and methodology.
 Sometimes you will follow multiple approaches/methodologies.
 
\begin{itemize}
\item  \textbf{Empirical vs. Non-empirical}: Empirical means to gather experiences about the reality and to order them in a kind of semantic. All steps will be documented, are comprehensible and can be repeated.
Non-empirical research focuses on understanding single topics, using common scientific knowledge in combination with theoretical scientific knowledge. It can be considered a theoretical research \cite{hans2005methoden}.

\item \textbf{Qualitative vs. Quantitative}: A qualitative research approach can be defined as exploratory research using methods such as observation, surveys or questionnaires. Qualitative methods are mostly used in social sciences to observe human behavior. Quantitative research focuses on the systematic empirical investigation of a topic or phenomena. Typically, computation techniques are used. Normally, quantities are \textbf{measured} such as data throughput, time or the amount of something.

\end{itemize}

\subsection{Research Methodology}

In this subsection we introduce some common research methodologies. The list is not comprehensive and the explanations given are not complete. \\

\begin{itemize}
\item \textbf{Conceptional Analysis:} You will investigate on a single question such as 'what is knowledge?' or 'what is a Continuous Delivery'? 
\item \textbf{Concept Implementation:} You will implement an approach you consider to be promising. It focuses on the implementation and its implementation details.
\item \textbf{Case Study:} It is an empirical research methodology, that implements and analyses a phenomenon considering its real-life context.
\item \textbf{Literature Review:} You conduct a structured literature review, categorizing, analyzing and comparing literature in your field of interest. 
\item \textbf{Simulation:} You simulate an algorithm or an approach. During the simulation you are able to control all the dependent variables (as well as the independent).
\item \textbf{Laboratory Experiment:} A laboratory experiment typically includes humans to test a specific design or prototype. You need to take care about the internal and external validity as well as the structure of your experiment and the sample method. The sample method has direct impact if your results are representative or not. Sometimes you just want to provide anecdotal evidence. Internal validity is typically good to control, external validity is challenging.
\item \textbf{Field Experiment:} Same as the laboratory experiment with one major difference: It takes place in the field. You typically are not able to control all the influencing factors, as in a laboratory experiments. Internal validity is difficult to control, while external validity can be considered high. 
\item \textbf{Questionnaire:} Empirical research method using items to gain knowledge about certain research questions or hypothesis. 
\item \textbf{Interview:} Empirical research methodology where you have to have a face to face interview. You ask questions about the topic of interest. Expert interviews are a common interview type.

\end{itemize}
