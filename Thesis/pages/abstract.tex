\chapter{\abstractname}

The abstract summarizes your research project and serves as an overview of the following sections of your proposals.
The reader should be able to instantly understand the problem and get an idea of how you are planning to solve it.

Ideally, an abstract covers the following aspects and is structured accordingly:

\begin{itemize}
	\item \textbf{Motivation/Objective:} Why are you going study the problem? 
	\item \textbf{Problem Statement:} What problem are your trying to solve? 
	\item \textbf{Proposed Solution:} How do you want to tackle the problem? 
	\item \textbf{Approach:} How will you conduct your research?
	\item \textbf{(Expected) Results:} What are the expected results of your research? 
	\item \textbf{Conclusion:}What are your conclusions? 
\end{itemize}

The motivation and objective can be completed by a preamble which introduces the problem domain and facilitates the decision whether the topic is interesting for the reader or not.
A motivation should answer the following questions: why now?
Materials and methods are part of the approach and describe how you accomplished your task.
The result answers the ''what?'' of your written work.
The conclusion summarizes your work.\\

\noindent \textbf{Note:} Do not use citations in the abstract! \\ 

%TODO: Abstract


